\documentclass[11pt, oneside]{ctexart}
\usepackage[a4paper,bindingoffset=0.2in,%
            left=1in,right=1in,top=1in,bottom=1in,%
            footskip=.25in]{geometry}
\usepackage{graphicx}
\usepackage{enumerate}
\usepackage{url}
\usepackage{hyperref}
\usepackage{pbox}
\usepackage{CJKutf8}
\usepackage{amsmath}
\usepackage{arydshln}
\usepackage{multirow}
\usepackage{multicol}

\begin{document}
\title{加密文件系统设计}
\author{高嘉蕊, 14307130345 \\ 计算机科学与技术学院}
\maketitle

\section{Server}
服务器端只相当于物理存储,没有目录,全部扁平化拉平。保证服务器无法得到除了加密之后的文件内容之外的任何信息。同时对服务器的行为进行时时监测。


\section{File System}
\subsection{文件系统整体设计}
在同一个文件系统中,为每个用户建立单独文件夹,对文件夹的创建目录或文件等操作真实在文件夹中操作,同时本地保存文件备份。基于Fuse重写各种命令保证保证用户只能回退到root,不能进入其他用户文件夹。用户分享过程中,为每一个分组单独创建文件夹。对每个分享组的用户而言,在自己的目录下也可以看到这个文件,但是实际文件的备份存储位置为分组中。

对每一个用户,生成一对用于RSA加密的密钥$(n_1, e_1, d_1)$,一个用于AES加密的密钥$k$,用户负责保存私钥$(n_1, d_1)$。对每一个分享组,生成三对用于RSA加密的密钥$(n_1, e_1, d_1), (n_2, e_2, d_2)$, $(n_3, e_3, d_3)$,一个用于AES加密的密钥$k$,分享组中有读权限的用户保存$(n_1, d_1), (n_2, d_2)$,有写权限的用户保存$(n_1, d_1), (n_3, d_3)$,
有读写权限的用户保存$(n_1, d_1), (n_2, d_2), (n_3, d_3)$。

在root目录中,保存四个文件/一个文件夹:
\begin{itemize}
\item 全部用户RSA公钥$(n_1, e_1)$的路径;
\item 全部用户/共享组的$(n_1, e_1)$加密后的AES密钥$K$;
\item 全部共享组RSA公钥$(n_1, e_1), (n_2, e_2), (n_3, e_3)$的路径;
\item 非法操作企图的同户名及其操作;
\item /key/ 用于保存全部实际RSA公钥。
\end{itemize}

每个用户目录中,保存一个文件:
\begin{itemize}
\item 全部共享文件的全部路径名与其映射的共享文件实际存储位置。
\end{itemize}

\subsection{加密过程}
\subsubsection{文件名加密}
用户使用公钥$(n_1, e_1)$加密文件全部路径,例如userA/path/file1.txt,作为上传服务器的名字,如果出现与服务器上的文件重名(可能性非常小),则要求用户重命名才可以上传。保证每个文件的路径名是不同的。

分享组使用公钥$(n_1, e_1)$加密。其余同上。

\subsubsection{文件内容加密}
使用AES加密,密钥为$k$。$k$无需用户保存,使用RSA加密用户/分享组公钥$(n_1, e_1)$对$k$加密,之后保存在root目录下的文件中即可。

\subsubsection{用户上传文件}\label{sec:user-upload}
\begin{itemize}
\item 用户用自己的私钥$(n_1, d_1)$解密得到root文件中自己的DES密钥$k$,用$k$对文件内容进行加密。
\item 计算加密后文件(部分文件)的hash值,同时把这个值用自己的私钥$(n_1, d_1)$签名,加到文件的header中(文件的最开始部分,不改inode)。
\item 对文件全部路径用公钥$(n_1, e_1)$加密,之后上传到文件服务器。如果文件服务器上出现重名,要求用户对文件重命名。
\item 把加密上传的文件同时备份在本地。
\end{itemize}

\subsubsection{用户读文件}\label{sec:user-read}
\begin{itemize}
\item 对想读的文件的全部路径用$(n_1, e_1)$加密,在server寻找对应文件下载下来。
\item 对文件进行验证,使用$(n_1, e_1)$对前一次写的签名进行解密,得到的字符串应当为文件(部分文件)的hash值。如果不不符,认为服务器或者其他原因文件被非法修改,则重新上传本地备份的文件到服务器。
\item 如果验证成功,则认为读取到正确文件。使用私钥$(n_1, d_1)$解密得到root文件中自己的DES密钥$k$,用$k$对文件解密。
\end{itemize}

\subsubsection{用户写文件}
\begin{itemize}
\item 首先按照Sec\ref{sec:user-read}读文件。
\item 完成写文件操作之后,按照Sec\ref{sec:user-upload}上传。
\item 更新本地加密的备份文件。
\end{itemize}

\subsubsection{用户分享文件}\label{sec:share}
下面以用户userA把test.txt分享给userB读权限,userC写权限,userD读写权限为例:
\begin{itemize}
\item userA选择test.txt为分享文件,则如果之前上述三个人没有共享过文件,则创新新文件夹group\_userA\_userB\_userC,同时生成对应的$7$个密钥,更新root中的文件。
\item 更新三个用户自己目录下存储的共享文件对应表,保证可以找到共享文件的实际存储位置。
\item userA对按照Sec\ref{sec:user-read}读取文件,并删除服务器上的对应文件。
\item userA使用分享组的$(n_2, d_2)$对分享组的DES密钥解密得到$k$,对test.txt加密。
\item 计算加密后test.txt(部分文件)的hash值,同时把这个值用分享组的另一对私钥$(n_3, d_3)$加密签名,加到test.txt的header中。
\item 删除A用户中原来的加密后的test.txt的备份,把新加密好的test.txt被分到路径group\_userA\_userB\_userC/test.txt中。
\item 对test.txt的新路径公钥$(n_1, e_1)$加密,之后上传到文件服务器。如果文件服务器上出现重名,要求用户对文件重命名。
\item userA把私钥$(n_1, d_1), (n_2, d_2)$告诉有读权限的userB,把$(n_1, d_1), (n_3, d_3)$告诉有写权限的userC,把$(n_1, d_1), (n_2, d_2), (n_3, d_3)$告诉有读写权限的userD。
\end{itemize}

\subsubsection{共享文件读写}
通过Sec\ref{sec:share}可以看到,用户需要有$(n_1, d_1)$才能读文件名字,同时再有$(n_2, d_2)$才能解密得到AES的$k$读文件,再得到$(n_3, d_3)$才能正确签名。为了让用户保存尽可能少的私钥信息,文件系统生成新的RSA密钥对,把用户需要保存的多个私钥加密存储在文件系统之中,用户只需要保存一个私钥。

注意在共享这里文件系统在:读文件、文件上传、备份文件更新之前都要进行签名检验。其余操作类似用户文件读写。

\section{完成需求}
由于服务器相当于物理存储,因此除了加密之后的文件内容外,得不到任何其他信息。

文件系统除了实现基本需求外,由于它只保存相关公钥、加密过的DES密钥、加密过的备份文件。所以,不仅恶意用户无法破坏文件系统,即使恶意用户破坏了文件系统,也只能得到其他用户的文件目录和文件夹名信息。

\section{存在问题}
\begin{itemize}
\item 用户需要保存的密钥太多,包括自己的私钥,以及每个分享组的私钥;
\item 因为想要恢复被服务器非法更改的文件,只能进行文件本地备份。
\end{itemize}

\end{document}